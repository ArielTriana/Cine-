\chapter{Enfoque Metodológico}

Para el desarrollo de este proyecto se aplicará la metodología ágil Extreme Programming (XP). 

Se seleccionó una metodología ágil debido a los beneficios de estas y la adaptabilidad a las características de nuestro problema. Una de sus ventajas es que facilita la planificación debido a la división de proyectos en sprints, siendo mas sencillo para el desarrollador abordar el proyecto en pequeñas fases con una duración y alcance determinados. El aumento de la implicación y de la motivación del equipo constituye otra ventaja, puesto que todos los miembros del equipo están informados del estado del proyecto. El cliente desde las primeras etapas del proyecto se le entrega un producto mínimo viable, por lo que este no debe esperar mucho tiempo para empezar a recuperar la inversión realizada. Las entregas continuas garantizan que tras la finalización de cada sprint el software se actualiza con nuevas modificaciones, mejora la calidad del producto debido a los testeos que se realizan después de la finalización de cada fase de desarrollo. La flexibilidad antes los cambios es otra de las características mas importantes ya que permite adaptación del proyecto a las nuevas necesidades o imprevistos que puedan surgir. 

Entre el grupo de metodologías ágiles se seleccionó la metodología XP de acuerdo a las características y ventajas de esta sobre nuestro problema.

Una de estas características que nos llevó a escoger esta metodología y no Scrum es debido a los requisitos del cliente. Se planean reuniones cada 15 días donde se entrega los avances del producto y un informe adjunto sobre el trabajo realizado en esos días, no se planean reuniones diarias, aunque  contamos con la disponibilidad del cliente para todos los encuentros en los que sea necesario debatir y valorar las sugerencias que brinde nuestro equipo y el cliente para lograr simplicidad, lo cual propicia la retroalimentación frecuente entre ambas partes, lo cual es una práctica de la metodología escogida (Cliente in-situ).

El trabajo se realizará en parejas lo cual constituye una principal característica la metodología XP, teniendo como gran ventaja la detección de errores conforme son introducidos en el código (inspecciones de código continuas), por consiguiente, la tasa de errores del producto final es más baja, los diseños son mejores y el tamaño del código menor (continua discusión de ideas de los programadores), varias personas entienden las diferentes partes sistema.

El listado de los requisitos del proyecto en cuestión se encuentra bien detallado, pero debido a la inexperiencia del personal, el desarrollo de este puede sufrir cambios por lo que el empleo de esta metodología facilitará la asimilación de estos. 

El personal tiene como principio hacer un software de calidad, en el que se apliquen los principios SOLID, DRY, KISS, YAGNI, una arquitectura que permita el desacoplamiento y extensibilidad, además que constituye una exigencia del cliente. De esta manera pondríamos en práctica la refactorización y el diseño simple.

Como característica del personal se permite que cualquier programador puede cambiar cualquier parte del código en cualquier momento motivando a la aparición de nuevas ideas por parte del personal. 

Cada uno de los motivos anteriores expuestos conllevó a la utilización de esta metodología ante el resto de metodologías ágiles en el proyecto asignado.
