\chapter{Introducción}

\section{Alcance del producto}

N/A

\section{Descripción general}

\subsection{Perspectiva del producto}

El producto busca controlar la venta de entradas de un cine denominado Cine+. Para esto el producto debe dar soporte tanto a la venta de entradas por taquillas como a la venta de entradas por internet, garantizándose la coordinación de las mismas.

\subsection{Funciones del Producto}

El producto es una aplicación web multiplataforma que permite la compra de entradas a través de su página web por parte de los clientes y calcula el costo de la entradas atendiendo a diferentes parámetros. Además da soporte a la inscripción de usuarios como socios del club de Cine+, así como los beneficios de los que estos pueden disfrutar. También el producto permite la anulación de las compras de entradas por parte de los usuarios y hace las actualizaciones pertinentes dada la anulación. El producto permite además que los gerentes del cine puedan actualizar el listado de películas y horarios disponibles y que estos sean mostrados en la web. Además estos pueden consultar las estadísticas de venta de entradas por diferentes parámetros. También se posibilita la visualización en una página web de las 10 películas sugeridas por uno de los gerentes del cine.

\subsection{Características de los usuarios}

Con el producto interactuarán tres tipos de usuarios: clientes, taquilleros y gerentes. Los gerentes dominan toda la información relacionada con la puesta en escena de las peliculas: horarios, precios, películas que se mostrarán, sugerencias, etc. Los taquilleros dominan como controlar el sistema de tal forma que la venta de entrada puedea mantenerse coordinada entre la venta física en la taquilla y la venta web. Los 3 tipos de usuarios están identificados tanto con dispositivos móviles y tabletas como con computadoras.

Los gerentes y taquilleros tienen el conocimiento necesario para operar este tipo de aplicación web realizando sus respectivas funciones. Los clientes conforman un grupo muy heterogéneo, algunos no tienen conocimiento con productos similares, ahí la necesidad de que la interfaz con la que interactúan sea cómoda.

\subsection{Restricciones Generales}

El cliente solicita que el sistema en cuestión sea una aplicación web, con las funcionalidades anteriormente mencionadas. Además informa que ya reservaron el hosting y dominio (www.cine+.com) de la página y la base de datos. El hosting tiene 4000 MB de almacenamiento.

El cliente pide que las páginas carguen en el tiempo recomendado por los expertos en posicionamiento SEO, para el posicionamiento en buscadores como Google o Bing. Además que el sitio tenga un flujo de navegación sencillo, y que la navegación no sobrepase el tercer nivel.

\section{Resumen del resto del documento}

En el Cap\'itulo 2 nos referiremos a los requerimientos espec\'ificos del producto, por lo que abordaremos sus requerimientos funcionales, no funcionales y de entorno. En el Cap\'itulo 3 abundaremos en las diferentes funcionalidades de la aplicaci\'on y se explicar\'a como ocurre la interacci\'on entre los diferentes tipos de usuarios y el producto.

El el Cap\'itulo 4 comentaremos sobre la metodolog\'ia seleccionada y daremos argumentos del porqu\'e de su elecci\'on. Adem\'as nos referiremos a los principios que seguimos para el desarrollo de la aplicaci\'on. En el Cap\'itulo 5 mencionaremos la arquitectura utilizada y expondremos los motivos de la elecci\'on de dicha arquitectura por encima de otras que tambi\'en fueron analizadas.

En el cap\'itulo 6 abordaremos tanto los patrones de visualizaci\'on de datos como los patrones de acceso a datos empleados en el desarrollo del producto. Finalmente en el cap\'itulo 7 estaremos viendo la modelaci\'on de la base de datos.